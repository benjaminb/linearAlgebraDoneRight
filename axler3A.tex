\documentclass{scrartcl}
\usepackage[T1]{fontenc}
\usepackage{ntheorem}

\usepackage{amsmath}
\usepackage{mathpazo}
\usepackage{graphicx}
\usepackage{mathrsfs}
\newcommand{\R}{\mathbb{R}}

\newcommand*{\vertbar}{\rule[-1ex]{0.5pt}{2.5ex}}
\newcommand*{\horzbar}{\rule[.5ex]{2.5ex}{0.5pt}}

\begin{document}
\title{Linear Algebra Done Right}
\subtitle{Section 3A}
\author{Selected solutions by Benjamin Basseri}
\date{}
\maketitle

\begin{enumerate}
	\item Suppose $b, c \in \R$. Define $T: \R^3 \rightarrow \R^2$ by 
$$T(x, y, z) = (2x-4y+3z+b, 6x + cxyz)$$

Show that $T$ is linear if and only if $b = c = 0$.

\textbf{Solution.} For the forward proof, assume $T$ is linear. Suppose we have two vectors in $\R^3, \mathbf{x} = (x, y, z)$ and $\mathbf{x}' = (x', y', z')$. Then by definition of $T$ and vector addition:
$$T(\mathbf{x} + \mathbf{x}') = T(x + x', y+y', z+z')$$
$$ = (2(x+x') - 4(y+y') + 3(z+z') + b, 6(x+x')+c(x+x')(y+y')(z+z'))$$

Since $T$ is linear, it is additive and we can also write $T(\mathbf{x} + \mathbf{x}')$ as:
\begin{align*}
	T(\mathbf{x} + \mathbf{x}') &= T(\mathbf{x}) + T(\mathbf{x}') &\text{linearity}\\
	&=(2x - 4y + 3z + b, 6x + cxyz)&\\
	& \quad  + (2x' - 4y' + 3z' + b, 6x' +cx'y'z')&\\
	&= \left(2(x+x') - 4(y+y') + 3(z+z') + 2b, 6(x+x') + c(xyz + x'y'z')\right)&\\
\end{align*}

By matching coordinates, we have $2b = b \implies b = 0$. We must also have $c(x+x')(y+y')(z+z') = c(xyz + x'y'z')$ for all $\mathbf{x}, \mathbf{x}' \in \R^3$, which implies $c=0$.

Since $T$ is linear it has homogeneity, so we can write:
$$\lambda T(\mathbf{x}) = \lambda (2x-4y+3z+b, 6x + cxyz) = T(\lambda \mathbf{x})$$

by distributivity of multiplication. 


For the reverse direction, assume $b = c = 0$. Then from the expansion of $T(\mathbf{x} + \mathbf{x}')$ above we see that the surviving terms in both coordinates satisfy linearity.

\end{enumerate}
\end{document}